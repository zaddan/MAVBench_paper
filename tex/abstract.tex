\begin{abstract}
Unmanned Aerial Vehicles (UAVs) are getting closer to becoming ubiquitous in everyday life. Among them, Micro Aerial Vehicles (MAVs) have seen an outburst of attention recently, specifically in the area with a demand for autonomy. A key challenge standing in the way of making MAVs autonomous is that researchers lack the comprehensive understanding of how performance, power, and computational bottlenecks affect MAV applications. MAVs must operate under a stringent power budget, which severely limits their flight endurance time. As such, there is a need for new tools, benchmarks, and methodologies to foster the systematic development of autonomous MAVs. In this paper, we introduce the ``MAVBench'' framework which consists of a closed-loop simulator and an end-to-end application benchmark suite. A closed-loop simulation platform is needed to probe and understand the intra-system (application data flow) and inter-system (system and environment) interactions in MAV applications to pinpoint bottlenecks and identify opportunities for hardware and software co-design and optimization. In addition to the simulator, MAVBench provides a benchmark suite, the first of its kind, consisting of a variety of MAV applications designed to enable computer architects to perform characterization and develop future aerial computing systems. Using our open source, end-to-end experimental platform, we uncover a hidden, and thus far unexpected compute to total system energy relationship in MAVs. Furthermore, we explore the role of compute by presenting three case studies targeting performance, energy and reliability. These studies confirm that an efficient system design can improve MAV's battery consumption by up to 1.8X. 
%life by up to 43\%.

\end{abstract}
